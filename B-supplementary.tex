\section*{\revise{Supplementary Material}}
\label{sec:SupplementaryMaterial}
\subsubsection*{\revise{Details of experiments in retrosplenial cortex.}}

\revise{Alexander and Nitz \cite{AlexanderNitz2015} investigated the spatial reference frame(s) to which retrosplenial cortical activity is anchored. In their experiments, six Long-Evans rats traversed a W-shaped track that occupied two distinct positions in the recording room, referred to as track positions $\alpha$ and $\beta$, which varied by recording session. The track was composed of two sets of action sequences depending on the direction of traversal (outbound or inbound trials). Outbound and inbound runs were made up of opposite turn sequences (left-right-left (LRL) and right-left-right (RLR), respectively. The manipulation of turn sequence, route, and track position allowed the assessment of neural sensitivity to the \textit{allocentric}, \textit{egocentric} and \textit{route-based} reference frames by comparing the observed firing patterns of electrophysiologically recorded neurons by the rat's position with respect to the route, the room, and the action being performed. In total, 243 neurons were recorded over 71 recording sessions, with measures taken to ensure maximum independence between the neurons that were recorded in each session. Each session had approximately 20 trials per condition (track position by direction of traversal). Additionally, Alexander and Nitz \cite{AlexanderNitz2015} recorded relevant behavioral data concurrently with neuronal firing patterns for each trial, including head direction (HD), position in X-Y coordinates (Pos), linear velocity (LV), and angular velocity (AV).}

\revise{Using the experimental data collected in these experiments, Rounds et al. \cite{Rounds2016,Rounds2018} used an evolutionary strategy to evolve a population of spiking neural networks (SNNs) that could replicate the functional, behavioral, and population responses observed in the electrophysiologically recorded data in response to the recorded behavioral metrics HD, Pos, LV, and AV. Each network contained 600 neurons (480 excitatory and 120 inhibitory Izhikevich neurons \cite{izhikevich2003simple}). Each trial consisted of 200 bins, each associated with a specific combination of these four inputs. The recorded values were encoded using cosine and Gaussian tuning curves that were subjected to a Poisson process to produce spiking inputs. The population was allowed to evolve over 50 generations, with convergence occurring by approximately the 20th generation. Synthetic neural activity was averaged across trials for each track position/traversal combination and then correlated with the 243 electrophysiologically recorded neuronal firing patterns. For each of the electrophysiologically recorded neurons, the synthetic neuron with the highest-correlated firing pattern was assigned as a match for that neuron. No duplicate matches were allowed - a neuron could be matched only once. Each SNN in the population was evaluated according to a fitness function that measured the sum of the highest correlations between neurons, with a penalty for overly high average maximum firing rates for the synthetic neurons to ensure a stable firing regime. The networks consistently converged to a fitness value of 105.93 $\pm$ 0.91 (arbitrary units), or an average correlation of Pearson's \textit{R} = 0.43 per neuron (high correlations by experimental standards). For further details on experimental methods, see \cite{AlexanderNitz2015,Rounds2016,Rounds2018}.}
