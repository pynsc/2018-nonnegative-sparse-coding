\section{Nonnegative sparse coding as a modern variant of the efficient coding hypothesis}

Early theories of efficient coding
\citep{Barlow1961,Attneave1954}
were developed based on the visual system.
Realizing that the set of natural scenes was much smaller
than the set of all possible images,
it was argued that the visual system should not 
waste  resources on processing arbitrary images.
Instead, the nervous system should use statistical knowledge
about its environment to represent the relevant input space 
as economically as possible.

Modern renditions of this theory
(such as sparse coding \citep{OlshausenField1996} and
\ac{ICA} \citep{BellSejnowski1997})
have refined the original hypotheses
by tying neuronal response properties 
to the statistics of the natural environment
(for a review, see \cite{SimoncelliOlshausen2001}).
These theories were shaped by two
fundamental empirical observations
about early visual cortex:
1) a neuron's \textbf{\ac{RF}} resembled a decomposition of the visual
stimulus into a series of local, largely independent feature components
(e.g., a 2-D Gabor function is basically a local approximation of the
directional spatial derivative of an image),
and 2) any individual neuron responded only sparsely to a small subset of
stimulus features (e.g., orientation or color at a particular spatial location).
Thus a neuron's \ac{RF} could be understood as a
sparse, low-dimensional embedding of high-dimensional input stimuli.
Such a representation is desirable from an energy-expenditure point of view,
since it allows the visual system to represent
any visual stimulus by activating only a small set of neurons,
while most neurons in the population remain silent.

Olshausen and Field \citep{OlshausenField1996} found that
linear sparse coding of natural images
yielded features qualitatively similar to
\acp{RF} of simple cells in \ac{V1},
thus giving empirically observed \acp{RF} an information-theoretic explanation.
However, as pointed out by Hoyer \citep{Hoyer2003}, sparse coding falls short
of providing a literal interpretation for \ac{V1} simple-cell behavior
for two reasons:
1) every neuron could be either positively or negatively active, and
2) the input to the neural network was typically double-signed,
whereas \ac{V1} neurons receive visual input from the \ac{LGN} 
in the form of separated, nonnegative ON and OFF channels.

Hoyer \citep{Hoyer2002,Hoyer2003} proposed \ac{NSC} as a way 
to transform Olshausen and Field's sparse coding 
from a relatively abstract model of image representation 
into a biologically plausible model of early visual cortex processing
by enforcing both input signal and neuronal activation to be nonnegative.
This seemingly simple change had remarkable consequences on the quality of the
sensory representation:
Whereas elementary image features in the standard sparse coding model could
``cancel each other out'' through subtractive interactions,
enforcing nonnegativity ensured that features combined additively,
much like the intuitive notion of combining parts to form a whole.
The resulting \emph{parts-based} representations resembled \acp{RF} in \ac{V1} more closely
than other \emph{holistic} representations,
and would later be shown to apply to other brain regions as well,
suggesting that parts-based representations are employed throughout cortex.

Inhibitory connections can be modeled in the same fashion,
by interpreting inhibitory weights as nonnegative synaptic conductances,
which not only
% Since neurons tend to release the same set of neurotransmitter at all of their
% terminals (Dale's principle), one would rarely have to use both excitatory and inhibitory weights
% Mixing of excitatory and inhibitory weights \emph{within a particular pre-post connection} 
% is rare, since neurons tend to release a single kind or class of neurotransmitters
% (Dale's principle).
% This is consistent with Dale's principle, which states that neurons
% tend to release a single kind or class of neurotransmitters.
% Modeling all synaptic weights as nonnegative synaptic conductances not only
preserves the parts-based quality of the encoding,
but also allows for more complicated connection types to be modeled
(e.g., \ac{V1} neurons receiving input from both excitatory ON
and inhibitory OFF cells in the \ac{LGN})
\citep{Hoyer2003}.
However, it is interesting to note that a more recent study has argued
that the nonnegativity constraint on the
synaptic weights might not be necessary to preserve the parts-based quality of
the encoding \citep{Liu2017}.

% Although inhibitory connections can possibly be interpreted as a way for parts to cancel, this does not equate to negative coding. In Hoyer's model, he interpreted some weights as E-E connections and others as I-E connections, but all were restricted to a nonnegative domain.