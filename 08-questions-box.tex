\section{Outstanding Questions Box}


% ELR: 2000 character limit with spaces. 
% Currently at 1885.

\begin{itemize}

\item That \ac{STDPH} can approximate \ac{NMF} under certain conditions has been proven for the single neuron case, but whether or not a  SNN with that form of \ac{STDPH} can converge to approximate \ac{NMF} is an open question. Can the proof be extended to apply to networks of neurons?

\item The application of \ac{NMF} to neuron populations is due to the fact that \ac{NMF} with sparsity constraints produces sparse, additive, parts-based representations of the features of an input space. It is established that many regions of the brain employ a sparse coding scheme, but the nature of representations and neuronal receptive fields are far less understood. Do parts-based representations appear in regions other than visual cortex that are also known for their sparsity, especially in higher cortical and subcortical regions?

\item \ac{NMF} produces nonnegative basis vectors that combine additively to represent an input space, which is consistent with the fact that neurons cannot have negative firing rates. However, \ac{STDP} can come in many forms, including inhibitory and neuromodulatory. Are there multiple kinds of dimensionality reduction implemented across the brain, determined by the statistical constraints placed upon the synapses of the region by the specific form of \ac{STDP} on the synapse?

\item Is it possible for a linear dimensionality reduction technique to generate a sufficiently rich and faithful representation of a high-dimensional input space that is practical and useful to the organism, or do more sophisticated forms of dimensionality reduction take place?

\item Does \ac{NSC} take place in every brain region, or just some? 

\item The connectivity in hard-to-reach brain regions is not well studied. Can weight matrices produced by \ac{STDPH} and \ac{NMF} with sparsity approximate the structure and connectivity of that observed in biological brains?

\end{itemize}