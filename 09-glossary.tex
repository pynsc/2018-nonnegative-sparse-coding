% Ok, so apparently the glossary isn't just a list of all acronyms
% Pick a few central terms that are worth explaining
% Limit to 450 words

\section{Glossary}

\begin{itemize}
% list alphabetically
% term in bold, then explain more than just long version of abbreviation

\item \textbf{Basis functions} A lower-dimensional set of linearly independent elements that can represent a high-dimensional input space given a weighted sum of these elements, where the weight of each element is defined by a separate hidden component.

% \item \textbf{Binary tournament selection} A selection strategy for producing new parents for the next generation in an evolutionary strategy. In tournament selection, $x$ individuals are chosen randomly, and then the individual in the subset with the highest fitness is chosen as a parent. This is repeated until the number of desired parents have been chosen. In binary tournament selection, $x = 2$ individuals are chosen at random from the population.

% \item \textbf{Crossover} In an evolutionary strategy, parents chosen to produce the next generation may undergo crossover. Each parent has a value associated with each parameter being evolved. To produce a child, the values associated with a parameter may be swapped at random (using a predetermined threshold) between the parents so that the child has a mixture of both parents' parameter values.

\item \textbf{Dimensionality reduction} The process of reducing the dimensionality of a space to the lowest possible space that encapsulates the variance of the original data via feature extraction. In the case of neuronal firing rate patterns, this means representing all possible firing rate patterns in the brain region using the smallest possible subset of the neurons.

\item \textbf{Factor analysis} A statistical method used to describe variability among observed, correlated variables in terms of a potentially lower number of unobserved, uncorrelated variables called factors (or latent variables).

% \item \textbf{Fitness} In evolutionary strategies, a single value computed with a user-defined function that represents how well an individual in a population meets the desired function that is being evolved.

% \item \textbf{Mutation} In an evolutionary strategy, parents chosen to produce the next generation may undergo mutation. Each parent has a value associated with each parameter being evolved. Each parameter may be randomly altered (according to a predetermined threshold) by selecting a new value from some distribution of values that lie within the established range to produce a new child individual for the next generation.

\item \textbf{Receptive field} The structure and boundaries of an individual neuron's pattern of response to various kinds of incoming stimuli.

\item \textbf{Spike-timing dependent plasticity} A Hebbian-inspired learning rule in which weight updates are computed based on the precise spike times of pre- and post-synaptic neurons that induce either long term potentiation or long term depression in the synapse, depending on whether the total pre-synaptic spike count preceded the total post-synaptic spike count, integrated over a critical temporal window.

\end{itemize}