% Ok, so apparently the glossary isn't just a list of all acronyms
% Pick a few central terms that are worth explaining
% Limit to 450 words

\section*{\revise{Glossary}}
\label{sec:Glossary}


\begin{itemize}
% list alphabetically
% term in bold, then explain more than just long version of abbreviation
\item \textbf{Allocentric reference frame} A spatial frame of reference that is defined with respect to a broader environment (e.g., one's location on a map). Hippocampal place cells are a textbook example of neurons that are anchored to an allocentric reference frame.

\item \textbf{Basis functions} A lower-dimensional set of linearly independent elements that can represent a high-dimensional input space given a weighted sum of these elements, where the weight of each element is defined by a separate hidden component.

% \item \textbf{Binary tournament selection} A selection strategy for producing new parents for the next generation in an evolutionary strategy. In tournament selection, $x$ individuals are chosen randomly, and then the individual in the subset with the highest fitness is chosen as a parent. This is repeated until the number of desired parents have been chosen. In binary tournament selection, $x = 2$ individuals are chosen at random from the population.

% \item \textbf{Crossover} In an evolutionary strategy, parents chosen to produce the next generation may undergo crossover. Each parent has a value associated with each parameter being evolved. To produce a child, the values associated with a parameter may be swapped at random (using a predetermined threshold) between the parents so that the child has a mixture of both parents' parameter values.

\item \textbf{Dimensionality reduction} The process of reducing the dimensionality of a space to the lowest possible space that encapsulates the variance of the original data via feature extraction. In the case of neuronal firing rate patterns, this means representing all possible firing rate patterns in the brain region using the smallest possible subset of the neurons.

\item \textbf{Efficient coding} TODO

% \item \textbf{Egocentric reference frame} \mikeNote{Really needed? not really important} A spatial frame of reference that is defined with respect to one's own perspective (e.g., taking a left turn is an action performed with respect to the egocentric reference frame).

% \item \textbf{Factor analysis} \mikeNote{Really needed? not really important, I'd rather strip it from the main text} A statistical method used to describe variability among observed, correlated variables in terms of a potentially lower number of unobserved, uncorrelated variables called factors (or latent variables).

% \item \textbf{Fitness} In evolutionary strategies, a single value computed with a user-defined function that represents how well an individual in a population meets the desired function that is being evolved.

\item \revise{\textbf{Holistic representation} Representation of a stimulus space that does not rely on explicit representations of stimulus component parts.
For example, a house might be represented by the visual system as a set of house
``templates''. Although visual information from individual house components
(e.g., front door, windows, roof, etc.) would of course be included in the house
representation, that information would be not be contained in representational
packets corresponding to the parse of the house into these features.
Instead, houses would be recognized ``all of a piece''.
}

\item \textbf{\Acf{ICA}} TODO

% \item \textbf{Mutation} In an evolutionary strategy, parents chosen to produce the next generation may undergo mutation. Each parent has a value associated with each parameter being evolved. Each parameter may be randomly altered (according to a predetermined threshold) by selecting a new value from some distribution of values that lie within the established range to produce a new child individual for the next generation.

\item \revise{\textbf{Parts-based representation} Representation of a stimulus space
in terms of explicit representations of stimulus component parts.
For example, a house might be decomposed by the visual system into a set of doors,
windows, a roof, etc. The resulting representation of the house would consist of
representations of these parts, somehow linked together.}

\item \textbf{\Acf{PCA}} TODO

\item \textbf{Population code} TODO
\mikeNote{This could be a relevant one}

\item \textbf{Receptive field} \mikeNote{Really needed? they're biologists} The structure and boundaries of an individual neuron's pattern of response to various kinds of incoming stimuli.

\item \textbf{Representational capacity} The ability to represent information depends on the number of recognizably different patterns of neuronal activity that can be generated in a useful time interval. This number---the representational capacity---is a fundamental measure of neural performance \cite{Laughlin2001}.

\item \textbf{Route-centric reference frame} A spatial frame of reference that is defined with respect to a planned path through a broader environment. For example, neurons in some parts of the brain fire for a particular location in a route, even if the route is repositioned or reoriented in the broader environment \cite{nitz2009parietal}.

\item \textbf{Sparse coding} TODO.

\item \textbf{Spike-timing dependent plasticity} \mikeNote{Really needed? they're biologists} A Hebbian-inspired learning rule in which weight updates are computed based on the precise spike times of pre- and post-synaptic neurons that induce either long term potentiation or long term depression in the synapse, depending on whether the total pre-synaptic spike count preceded the total post-synaptic spike count, integrated over a critical temporal window.

\end{itemize}