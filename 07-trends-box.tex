\section{Trends Box}

% 900 character limit with spaces - currently at 879 chars in word
%----

% The Trends box is a short collection of bullet point statements (3-5) that concisely convey to the reader the recent advances in the area, including emerging concepts and/or distinctions, that constitute a main motivation for the discussion developed in the article. 

% \begin{itemize}
% \item As the box aims to focus on recent developments, conclusions and future directions should instead be discussed in the Concluding Remarks section and/or the Outstanding Questions box.
% \item The text in the Trends box may not exceed 900 characters, including spaces. 
% \item The Trends Box is not called out in the text.
% \item The Trends Box does not count towards the total number of allowed display elements in the manuscript (limit of 4).
% \item When submitting the manuscript files via Editorial Manager, please upload the Trends Box as a separate word file using the designated heading.
% \end{itemize}

\begin{itemize}
\item Sparse coding models were developed to describe sensory coding in the visual cortex, but evidence for nonnegative, sparse coding schemes exists for many brain regions.

\item  Current thought conceptualizes neuron function as highly selective for specific inputs for the purpose of performing specialized behaviors, but neurons may instead generalize by performing dimensionality reduction on their inputs in order to represent as much information about incoming stimuli as possible, and their functional behavior may simply be an emergent phenomenon resulting from this computation.

\item New evidence suggests that certain forms of spike-timing dependent plasticity, in conjunction with lateral inhibition, may approximate a form of dimensionality reduction that we call nonnegative sparse coding (\ac{NSC}), which may be a common computation performed throughout the brain.

% \item Recent computational studies suggest that nonnegative sparse codes
% might be more prevalent in the brain than previously thought. 
% \mikeNote{Wooooords}
% \emilyNote{aren't they already considered prevalent? not sure I like this point. Also it seems a lot like the second point.}

% \item Originally developed to describe sensory coding in visual cortex,
% we found evidence for nonnegative sparse coding all over the brain.

% \item We think that NSC can be implemented with spike-timing dependent
% plasticity and homeostasis.

\end{itemize}





% Recent advances in high-dimensional statistics and methods for analyzing brain data
% make this an ideal time for
% new discoveries and theories about
% the statistical properties of information processing in the brain.


% Similarly, the brain itself is a vastly complex nonlinear, highly-interconnected network and neuroscience requires tractable, generalizable models for these inherently high-dimensional neural systems.