\section*{Glossary}
\label{sec:Glossary}

\begin{itemize}
\item \textbf{Allocentric reference frame} A spatial frame of reference that is defined with respect to a broader environment (e.g., one's location on a map). Hippocampal place cells are a textbook example of neurons that are anchored to an allocentric reference frame.

\item \textbf{Basis functions} A lower-dimensional set of linearly independent elements that can represent a high-dimensional input space given a weighted sum of these elements, where the weight of each element is defined by a separate hidden component. For example, according to Fourier analysis, sine and cosine are basis functions for the space of all continuous periodic functions.

\item \textbf{Dimensionality reduction} The process of reducing the dimensionality of a space to the lowest possible space that encapsulates the variance of the original data via feature extraction. In the case of neuronal firing rate patterns, this means representing all possible firing rate patterns in the brain region using the smallest possible subset of the neurons.

\item \textbf{Efficient coding} A theoretical model of sensory coding in the brain based on information theory \cite{Barlow1961,Attneave1954,Linsker1990}. The efficient coding hypothesis posits that sensory pathways can be understood as communication channels where neuronal spiking aims to maximize available channel capacity by minimizing the redundancy between representational units.

\item \textbf{Holistic representation} Representation of a stimulus space that does not rely on explicit representations of stimulus component parts.
For example, a house might be represented by the visual system as a set of house
`templates'. Although visual information from individual house components
(e.g., front door, windows, roof, etc.) would of course be included in the house
representation, that information would be not be contained in representational
packets corresponding to the parse of the house into these features.
Instead, houses would be recognized `all of a piece'.

\item \textbf{\Acf{ICA}} A computational method for decomposing multivariate data into additive components by assuming that the components are non-Gaussian signals and statistically independent from each other. Independent components differ from decorrelated components by the fact that the minimization includes higher-order and not only second-order statistics. A simple application of \ac{ICA} is the `cocktail party problem', where the underlying speech signals are separated from a sample data consisting of people talking simultaneously in a room.

\item \textbf{\Acf{NMF}} A computational method for decomposing multivariate data into additive components by constraining the components to be nonnegative. This constraint results in a parts-based representation, because it only allows additive, and not subtractive, combinations of subcomponents.

\item \textbf{Parts-based representation} Representation of a stimulus space
in terms of explicit representations of stimulus component parts.
For example, a house might be decomposed by the visual system into a set of doors,
windows, a roof, etc. The resulting representation of the house would consist of
representations of these parts, somehow linked together.

\item \textbf{\Acf{PCA}} A computational method for decomposing multivariate data into linearly uncorrelated components. \Ac{PCA} identifies an ordered set of orthogonal directions that captures the greatest variance in the data \cite{CunninghamYu2014}.

\item \textbf{Representational capacity} The number of recognizably different patterns of neuronal activity that a population of neurons can generate in a useful time interval \cite{Laughlin2001}.

\item \textbf{Route-centric reference frame} A spatial frame of reference that is defined with respect to a planned path through a broader environment. For example, neurons in some parts of the brain fire for a particular location in a route, even if the route is repositioned or reoriented in the broader environment \cite{nitz2009parietal}.

\item \textbf{Sparse coding} A population coding scheme where activity is represented by the strong activation of a relatively small set of neurons. Sparse coding can be described as a trade-off between the benefits and drawbacks of dense and local codes \cite{Foldiak1990}.

\item \textbf{Spike-timing dependent plasticity (STDP)} A Hebbian-inspired learning rule in which weight updates are computed based on the precise spike times of pre- and post-synaptic neurons that induce either long term potentiation or long term depression in the synapse, depending on whether the total pre-synaptic spike count preceded the total post-synaptic spike count, integrated over a critical temporal window.

\end{itemize}